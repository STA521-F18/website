\PassOptionsToPackage{unicode=true}{hyperref} % options for packages loaded elsewhere
\PassOptionsToPackage{hyphens}{url}
%
\documentclass[ignorenonframetext,]{beamer}
\usepackage{pgfpages}
\setbeamertemplate{caption}[numbered]
\setbeamertemplate{caption label separator}{: }
\setbeamercolor{caption name}{fg=normal text.fg}
\beamertemplatenavigationsymbolsempty
\usepackage{lmodern}
\usepackage{amssymb,amsmath}
\usepackage{ifxetex,ifluatex}
\usepackage{fixltx2e} % provides \textsubscript
\ifnum 0\ifxetex 1\fi\ifluatex 1\fi=0 % if pdftex
  \usepackage[T1]{fontenc}
  \usepackage[utf8]{inputenc}
  \usepackage{textcomp} % provides euro and other symbols
\else % if luatex or xelatex
  \usepackage{unicode-math}
  \defaultfontfeatures{Ligatures=TeX,Scale=MatchLowercase}
\fi
% use upquote if available, for straight quotes in verbatim environments
\IfFileExists{upquote.sty}{\usepackage{upquote}}{}
% use microtype if available
\IfFileExists{microtype.sty}{%
\usepackage[]{microtype}
\UseMicrotypeSet[protrusion]{basicmath} % disable protrusion for tt fonts
}{}
\IfFileExists{parskip.sty}{%
\usepackage{parskip}
}{% else
\setlength{\parindent}{0pt}
\setlength{\parskip}{6pt plus 2pt minus 1pt}
}
\usepackage{hyperref}
\hypersetup{
            pdftitle={Introduction to Modern Regression and Predictive Modeling},
            pdfauthor={Merlise Clyde},
            pdfborder={0 0 0},
            breaklinks=true}
\urlstyle{same}  % don't use monospace font for urls
\newif\ifbibliography
\usepackage{longtable,booktabs}
\usepackage{caption}
% These lines are needed to make table captions work with longtable:
\makeatletter
\def\fnum@table{\tablename~\thetable}
\makeatother
\usepackage{graphicx,grffile}
\makeatletter
\def\maxwidth{\ifdim\Gin@nat@width>\linewidth\linewidth\else\Gin@nat@width\fi}
\def\maxheight{\ifdim\Gin@nat@height>\textheight\textheight\else\Gin@nat@height\fi}
\makeatother
% Scale images if necessary, so that they will not overflow the page
% margins by default, and it is still possible to overwrite the defaults
% using explicit options in \includegraphics[width, height, ...]{}
\setkeys{Gin}{width=\maxwidth,height=\maxheight,keepaspectratio}
% Prevent slide breaks in the middle of a paragraph:
\widowpenalties 1 10000
\raggedbottom
\setbeamertemplate{part page}{
\centering
\begin{beamercolorbox}[sep=16pt,center]{part title}
  \usebeamerfont{part title}\insertpart\par
\end{beamercolorbox}
}
\setbeamertemplate{section page}{
\centering
\begin{beamercolorbox}[sep=12pt,center]{part title}
  \usebeamerfont{section title}\insertsection\par
\end{beamercolorbox}
}
\setbeamertemplate{subsection page}{
\centering
\begin{beamercolorbox}[sep=8pt,center]{part title}
  \usebeamerfont{subsection title}\insertsubsection\par
\end{beamercolorbox}
}
\AtBeginPart{
  \frame{\partpage}
}
\AtBeginSection{
  \ifbibliography
  \else
    \frame{\sectionpage}
  \fi
}
\AtBeginSubsection{
  \frame{\subsectionpage}
}
\setlength{\emergencystretch}{3em}  % prevent overfull lines
\providecommand{\tightlist}{%
  \setlength{\itemsep}{0pt}\setlength{\parskip}{0pt}}
\setcounter{secnumdepth}{0}

% set default figure placement to htbp
\makeatletter
\def\fps@figure{htbp}
\makeatother


\title{Introduction to Modern Regression and Predictive Modeling}
\author{Merlise Clyde}
\date{8/29/2018}

\begin{document}
\frame{\titlepage}

\begin{frame}{Coordinates}
\protect\hypertarget{coordinates}{}

\begin{itemize}[<+->]
\tightlist
\item
  Instructor: Merlise Clyde
\item
  TAs:

  \begin{itemize}[<+->]
  \tightlist
  \item
    Jiurui Tang
  \item
    Abbas Zaidi
  \end{itemize}
\item
  Course Websites:

  \begin{itemize}[<+->]
  \tightlist
  \item
    Main \url{http://stat.duke.edu/courses/Fall18/sta521}
  \item
    Sakai \url{https://sakai.duke.edu/portal/site/sta521-f18}
  \item
    Github \url{https://github.com/STA521-F18}
  \end{itemize}
\end{itemize}

\end{frame}

\begin{frame}{Grading}
\protect\hypertarget{grading}{}

\begin{longtable}[]{@{}ll@{}}
\toprule
Component & Percentage\tabularnewline
\midrule
\endhead
Participation & 5\%\tabularnewline
Homework & 25\%\tabularnewline
Midterm 1 & 20\%\tabularnewline
Midterm 2 & 20\%\tabularnewline
Data Analysis Part I & 15\%\tabularnewline
Data Analysis Part II & 15\%\tabularnewline
\bottomrule
\end{longtable}

\end{frame}

\begin{frame}{Groups}
\protect\hypertarget{groups}{}

\begin{itemize}[<+->]
\item
  Team based data analysis assignments

  \begin{itemize}[<+->]
  \tightlist
  \item
    Roughly weekly assignments
  \item
    10 - 20 hours of work each
  \item
    Peer review at the end
  \end{itemize}
\item
  Periodic individual assignments for concepts/theory
\item
  Expectations and roles

  \begin{itemize}[<+->]
  \tightlist
  \item
    Everyone is expected to contribute equally
  \item
    Everyone is expected to understand \emph{all} code turned in
  \item
    Individual contribution evaluated by peer assessment
  \item
    You may help each other, but submitted work must be your own
  \end{itemize}
\end{itemize}

\end{frame}

\begin{frame}{Policies}
\protect\hypertarget{policies}{}

\begin{itemize}[<+->]
\tightlist
\item
  Duke Community Standard

  \begin{itemize}[<+->]
  \tightlist
  \item
    I will not lie, cheat, or steal in my academic endeavors
  \item
    I will conduct myself honorably in all of my endeavors; and
  \item
    I will act if the standard is compromised
  \end{itemize}
\item
  Plagiarism

  \begin{itemize}[<+->]
  \tightlist
  \item
    Use online resources (Stack-exchange, etc) but make sure to cite
    them (code or theory)
  \item
    No direct code sharing between groups / individuals
  \end{itemize}
\item
  Coding Homework

  \begin{itemize}[<+->]
  \tightlist
  \item
    Group based, everyone is equally responsible
  \end{itemize}
\item
  Late Homework Policy:

  \begin{itemize}[<+->]
  \tightlist
  \item
    No Late HW
  \item
    Drop the lowest score
  \end{itemize}
\item
  2 In-Class Midterms
\end{itemize}

\end{frame}

\begin{frame}{Reproducible Research / Data Analysis}
\protect\hypertarget{reproducible-research-data-analysis}{}

\begin{itemize}[<+->]
\item
  R + RStudio + JAGS
\item
  Rmarkdown/knitr
\item
  Git + github
\end{itemize}

\end{frame}

\begin{frame}{For Friday}
\protect\hypertarget{for-friday}{}

\begin{itemize}[<+->]
\tightlist
\item
  Install recommended
  \href{http://www2.stat.duke.edu/courses/Spring17/sta521/resources/}{software}

  \begin{itemize}[<+->]
  \tightlist
  \item
    \href{http://cran.r-project.org}{R}
  \item
    \href{https://www.rstudio.com/products/rstudio/download/}{Rstudio}
  \item
    \href{http://mcmc-jags.sourceforge.net}{JAGS}
  \end{itemize}
\item
  \href{https://www.codeschool.com/courses/try-r}{Try R Code School} if
  you are new to R
\item
  Create a \href{http://github.com}{github account} (if you do not have
  one already)
\item
  Complete the course survey (email link next week)
\end{itemize}

\end{frame}

\begin{frame}{Data Science}
\protect\hypertarget{data-science}{}

\includegraphics{http://www.ibm.com/developerworks/jp/opensource/library/os-datascience/figure1.png}

See (Bin Yu's IMS Presidential Address
2014){[}\url{http://bulletin.imstat.org/2014/10/ims-presidential-address-let-us-own-data-science/}{]}

\end{frame}

\begin{frame}{Modern Regression \& Predictive Modelling}
\protect\hypertarget{modern-regression-predictive-modelling}{}

\begin{itemize}[<+->]
\tightlist
\item
  Response variable \(Y_i\)
\item
  Inputs \(X_i\) (vector)
\item
  Goals:

  \begin{itemize}[<+->]
  \tightlist
  \item
    learn a model to \textbf{predict} \(Y_i\) given \(X_i\) at new
    inputs \(X_i\)\\
  \item
    understand relationship between \(X_i\) and \(Y_i\)
    (\textbf{inference})
  \item
    \(E[Y_i] = f(X_i)\) learn regression function \(f(X)\)
  \item
    Model Based Statistical Learning
  \end{itemize}
\end{itemize}

\end{frame}

\begin{frame}[fragile]{Course Expectations}
\protect\hypertarget{course-expectations}{}

\begin{itemize}[<+->]
\tightlist
\item
  Expect to deal with simple to increasingly messy data (real world)
\item
  Writing \texttt{R} and \texttt{JAGS} code that is reproducible
\item
  self-documented code/solutions using \texttt{Rmarkdown}
\item
  use of version control (git) for team based reproducible coding
\item
  interpretation of results for non-statisticians
\end{itemize}

\end{frame}

\begin{frame}{Course Topics}
\protect\hypertarget{course-topics}{}

\begin{itemize}[<+->]
\tightlist
\item
  Visualization and Exploratory Data Analysis
\item
  Linear Regression
\item
  Diagnostics and model checking
\item
  Predictive Distributions
\item
  Model Selection including variable selection, variable
  transformations, distribution choices
\item
  Model Uncertainty (Bayesian Model Averaging and other Ensemble
  Methods)
\item
  Bayesian Shrinkage and Penalized Likelihood Estimation (Ridge
  Regression/ LASSO/ Horseshoe )
\item
  Robust Estimation
\item
  Classification and Regression Trees, Random Forests, Boosting,
  Bayesian Additive Regression Trees
\item
  Other Topics: Nonparametric Regression, Time Series, Neural Networks
\end{itemize}

\end{frame}

\begin{frame}{Themes}
\protect\hypertarget{themes}{}

\begin{itemize}[<+->]
\tightlist
\item
  Interpretability versus predictive performance
\item
  Bias-Variance Trade-off
\item
  In sample versus out-of-sample
\item
  point estimates versus uncertainty quantification
\item
  exact analysis versus approximation (computational scaling)
\item
  understanding structure of data (relationships)
\item
  Bayesian versus Frequentist ?
\end{itemize}

\end{frame}

\begin{frame}{Tradeoffs\ldots{}}
\protect\hypertarget{tradeoffs}{}

\includegraphics{https://larspsyll.files.wordpress.com/2015/09/quote-all-models-are-wrong-but-some-are-useful-george-e-p-box-53-42-27.jpg}

\end{frame}

\begin{frame}{Frequenstist \& Bayes}
\protect\hypertarget{frequenstist-bayes}{}

\begin{itemize}[<+->]
\tightlist
\item
  Likelihood Based inference

  \begin{itemize}[<+->]
  \tightlist
  \item
    Sampling model \[Y_i \sim f(y_i \mid \theta)\]
  \item
    Likelihood \[L(\theta) \propto \prod_i f(Y_i \mid \theta)\]
  \item
    MLE of \(\theta\)
  \end{itemize}
\item
  Bayes

  \begin{itemize}[<+->]
  \tightlist
  \item
    prior distribution \(p(\theta)\) describes prior uncertainty about
    \(\theta\)
  \item
    posterior distribution
    \[p(\theta \mid data) \propto L(\theta)p(\theta)\]
  \item
    uncertainty after seeing data
  \end{itemize}
\end{itemize}

\end{frame}

\begin{frame}{Got Data?}
\protect\hypertarget{got-data}{}

\includegraphics{https://media.licdn.com/media-proxy/ext?w=800\&h=800\&hash=sfmK8PW\%2BTbPHupf8ExHOszMCVRg\%3D\&ora=1\%2CaFBCTXdkRmpGL2lvQUFBPQ\%2CxAVta9Er0Vinkhwfjw8177yE41y87UNCVordEGXyD3u0qYrdf3a_KpLcKrbzuVoeKiQclABkefL5FjPnD8G4I47rKY8ngsPmJo24ZxUBbFI8lWxI}

\end{frame}

\begin{frame}{Philosophy}
\protect\hypertarget{philosophy}{}

\begin{itemize}[<+->]
\item
  for many problems Frequentist and Bayesian methods will give similar
  answers (more a matter of taste in interpretation)

  \begin{itemize}[<+->]
  \tightlist
  \item
    For small problems, Bayesian methods allow us to incorporate prior
    information which provides better calibrated answers and better
    measure of uncertainty
  \item
    for problems with complex designs and/or missing data Bayesian
    methods are often better easier to implement (do not need to rely on
    asymptotics)
  \end{itemize}
\item
  For problems involving hypothesis testing or model selection
  Frequentist and Bayesian methods can be strikingly different.
\item
  Frequentist methods often faster (particularly with ``big data'') so
  great for exploratory analysis and for building a \emph{data-sense}
\item
  Bayesian methods sit on top of Frequentist Likelihood
\item
  Important to understand advantages and problems of each perspective!
\end{itemize}

\end{frame}

\begin{frame}{Statistical and Machine Learning ?}
\protect\hypertarget{statistical-and-machine-learning}{}

\includegraphics{https://imgs.xkcd.com/comics/machine_learning.png}

\end{frame}

\begin{frame}{Ovarian Cancer Risk Prediction}
\protect\hypertarget{ovarian-cancer-risk-prediction}{}

\begin{itemize}[<+->]
\item
  Binary Outcome (Cancer/Control)

  \begin{itemize}[<+->]
  \tightlist
  \item
    17 established SNPS (genetic markers)
  \item
    other risk factors (age, family history, oral contraceptive use,
    number of pregnancies, aspirin, \ldots{})
  \end{itemize}
\item
  Case - Control design\\
\item
  variability across study sites (random effects)
\item
  80\% subjects had at least one variable with missing data
\item
  Missing at random versus missing not-at-random
\item
  Focus is on prediction, but still need an interpretable model
\item
  EDA, Model Building, and Predictive Checking crucial!
\end{itemize}

\end{frame}

\end{document}
